% Options for packages loaded elsewhere
\PassOptionsToPackage{unicode}{hyperref}
\PassOptionsToPackage{hyphens}{url}
%
\documentclass[
]{article}
\title{README: GMBA SDG 15.4.1 Calculator}
\author{Amina Ly}
\date{11/23/2021}

\usepackage{amsmath,amssymb}
\usepackage{lmodern}
\usepackage{iftex}
\ifPDFTeX
  \usepackage[T1]{fontenc}
  \usepackage[utf8]{inputenc}
  \usepackage{textcomp} % provide euro and other symbols
\else % if luatex or xetex
  \usepackage{unicode-math}
  \defaultfontfeatures{Scale=MatchLowercase}
  \defaultfontfeatures[\rmfamily]{Ligatures=TeX,Scale=1}
\fi
% Use upquote if available, for straight quotes in verbatim environments
\IfFileExists{upquote.sty}{\usepackage{upquote}}{}
\IfFileExists{microtype.sty}{% use microtype if available
  \usepackage[]{microtype}
  \UseMicrotypeSet[protrusion]{basicmath} % disable protrusion for tt fonts
}{}
\makeatletter
\@ifundefined{KOMAClassName}{% if non-KOMA class
  \IfFileExists{parskip.sty}{%
    \usepackage{parskip}
  }{% else
    \setlength{\parindent}{0pt}
    \setlength{\parskip}{6pt plus 2pt minus 1pt}}
}{% if KOMA class
  \KOMAoptions{parskip=half}}
\makeatother
\usepackage{xcolor}
\IfFileExists{xurl.sty}{\usepackage{xurl}}{} % add URL line breaks if available
\IfFileExists{bookmark.sty}{\usepackage{bookmark}}{\usepackage{hyperref}}
\hypersetup{
  pdftitle={README: GMBA SDG 15.4.1 Calculator},
  pdfauthor={Amina Ly},
  hidelinks,
  pdfcreator={LaTeX via pandoc}}
\urlstyle{same} % disable monospaced font for URLs
\usepackage[margin=1in]{geometry}
\usepackage{graphicx}
\makeatletter
\def\maxwidth{\ifdim\Gin@nat@width>\linewidth\linewidth\else\Gin@nat@width\fi}
\def\maxheight{\ifdim\Gin@nat@height>\textheight\textheight\else\Gin@nat@height\fi}
\makeatother
% Scale images if necessary, so that they will not overflow the page
% margins by default, and it is still possible to overwrite the defaults
% using explicit options in \includegraphics[width, height, ...]{}
\setkeys{Gin}{width=\maxwidth,height=\maxheight,keepaspectratio}
% Set default figure placement to htbp
\makeatletter
\def\fps@figure{htbp}
\makeatother
\setlength{\emergencystretch}{3em} % prevent overfull lines
\providecommand{\tightlist}{%
  \setlength{\itemsep}{0pt}\setlength{\parskip}{0pt}}
\setcounter{secnumdepth}{-\maxdimen} % remove section numbering
\ifLuaTeX
  \usepackage{selnolig}  % disable illegal ligatures
\fi

\begin{document}
\maketitle

\hypertarget{purpose}{%
\subsection{Purpose:}\label{purpose}}

This project contains a set of scripts used to calculate mountain
biodiversity utilizing the GMBA mountain inventory V2.0. Contents of
this project are aimed at improving reporting progress towards SDG
15.4.1 by disaggregating calculations to the mountain level

\hypertarget{setup}{%
\subsection{Setup:}\label{setup}}

The results of this analysis are reproducible using only this repo, and
all r required input data files. To do so:

\begin{enumerate}
\def\labelenumi{\arabic{enumi}.}
\tightlist
\item
  Clone this repository
\item
  Download the required data files (request access from
  \textbf{\_\_\_\_})
\item
  Run the desired analysis (see below for details)
\end{enumerate}

While most input files are static for any analysis, some may be updated
annually Please refer to the file Input\_File\_Details in the home
directory of this project for more detailed information on the required
data files.

\hypertarget{included-analyses}{%
\subsection{Included Analyses:}\label{included-analyses}}

\emph{Detailed instructions on how to run these analyses can be found in
`Instructions to Prepare and Run Code.docx'}

\hypertarget{intersection_official.r}{%
\subsubsection{Intersection\_Official.R
:}\label{intersection_official.r}}

Calculate the \% coverage of mountainous Key Biodiversity Areas (KBAs)
by Protected Areas (PAs) using the official SDG method. For any KBA that
overlaps with a GMBA polygon, the \emph{full} KBA is overlapped with all
PAs to determine the \% coverage. For KBAs that increase in size, each
additional entry is the additional overlap. Results are \emph{not}
cumulative. Adapted from the Birdlife International script:
(\url{https://github.com/BirdLifeInternational/kba-overlap})

\hypertarget{output}{%
\paragraph{Output:}\label{output}}

\begin{itemize}
\tightlist
\item
  \textbf{/results/finaltab\_official\_`YEAR\_RUN'.csv''} : table with
  \% added coverage for each KBA
\item
  \textbf{/results/results\_country/\ldots{}} : folder with
  country-specific results
\end{itemize}

\hypertarget{intersection_gmba.r}{%
\subsubsection{Intersection\_GMBA.R}\label{intersection_gmba.r}}

Calculate the \% coverage

\hypertarget{output-1}{%
\paragraph{Output:}\label{output-1}}

\begin{itemize}
\tightlist
\item
  \textbf{/results/finaltab\_gmba\_`YEAR\_RUN'.csv} : table with \%
  coverage of each mountainous area of each KBA as identified by the
  GMBA polygons
\item
  \textbf{/results/results\_mt/\ldots{}} : folder with mountain
  range-specific results
\end{itemize}

\hypertarget{additional-scripts}{%
\subsection{Additional Scripts:}\label{additional-scripts}}

\hypertarget{data_prep_for_visuals.r}{%
\subsubsection{Data\_Prep\_for\_Visuals.R}\label{data_prep_for_visuals.r}}

To simplify aggregation and translate the data to show cumulative
coverage for every year, this script reads in the results of a previous
run (updated by user in section 1.2), and outputs multiple aggregation
files.

\hypertarget{output-2}{%
\paragraph{Output:}\label{output-2}}

\begin{itemize}
\tightlist
\item
  \textbf{results/finaltab\_cumulative.csv} : cumulative coverage by KBA
  for every year starting with the minimum year in the result file. Each
  row is the total coverage for a KBA in that year.
\item
  \textbf{results/finaltab\_cumulative\_country.csv} : aggregated to the
  country level \textbf{by total area}
\item
  \textbf{results/finaltab\_cumulative\_mtrange.csv} : aggregated to the
  mountain range \textbf{by total area}
\item
  \textbf{results/finaltab\_cumulative\_mtrange\_country.csv} :
  aggregated to the mountain range by country \textbf{by total area}
\item
  \textbf{results/finaltab\_cumulative\_country\_avg.csv} : aggregated
  to get the average coverage of KBAs within a country
\item
  \textbf{results/finaltab\_cumulative\_mtrange\_avg.csv} : aggregated
  to get the average coverage of KBAs within a mountain range
\item
  \textbf{results/finaltab\_cumulative\_mtrange\_country\_avg.csv} :
  aggregated to get the average coverage of KBAs within a country
\item
  \textbf{results/finaltab\_cumulative\_parentrange.csv} : aggregated to
  the intermediate range \textbf{by total area}
\item
  \textbf{results/finaltab\_cumulative\_parentrange\_country.csv} :
  aggregated to the intermediate range \textbf{by total area}
\end{itemize}

\hypertarget{contributors}{%
\subsection{Contributors:}\label{contributors}}

Amina Ly, Stanford University 2021

Based on Code by based on code by Ash Simkins \& Lizzie Pearmain, March
2020

\end{document}
